\documentclass{homework}
\newcommand{\hwname}{Zooey Nguyen}
\newcommand{\hwemail}{zooeyn@ucla.edu}
\newcommand{\hwclass}{Math 131A}
\newcommand{\hwtype}{Homework}
\newcommand{\hwnum}{1}
\begin{document}
\maketitle


\question
By Corollary 2.3 we know that any rational solution of this equation must be an integer that divides the bias $c_0$. Since $c_0 = -1$ we have only the options $\pm 1$, so let's check both.
\begin{align*}
    P_1 &: (1)^4 - 2(1)^3 + 3(1)^2 + 5(1) - 1   \\
        &=	1 - 2 + 3 + 5 - 1	\\
        &\ne 0  \\
    P_{-1} &: (-1)^4 - 2(-1)^3 + 3(-1)^2 + 5(-1) - 1   \\
        &=	1 + 2 + 3 - 5 - 1	\\
        &= 0  \\
\end{align*}
Thus the only rational solution to this is \fbox{-1}.


\question
Proof of the properties of a field.

\begin{enumerate}
    \item $a+c = b+c \Rightarrow a=b$. Assume $a+c = b+c$.
        \begin{align*}
            a   &=  a+0	\\
            a   &=	a + (c + (-c))	\\
            a   &=	(a + c) + (-c)	\\
            a   &=	(b + c) + (-c)	\\
            a   &=	b + (c + (-c))	\\
            a   &=	b + 0	\\
            a   &=  b
        \end{align*}
    \item $a \vdot 0 = 0$.
        \begin{align*}
            a \vdot 0	&=	a(0+0)	\\
            a \vdot 0    &=  a \vdot 0 + a \vdot 0   \\
            a \vdot 0 - a \vdot 0 &= a \vdot 0 + (a \vdot 0 - a \vdot 0)  \\
            0   &=  a \vdot 0 + 0   \\
            0   &=  a \vdot 0
        \end{align*}
    \item $(-a) b = -(ab)$
        \begin{align*}
            (-a) b	&=	(-a) b + 0	\\
            (-a) b	&=	(-a) b + (ab + -(ab))	\\
            (-a) b	&=	(-a) b + ab + -(ab)	\\
            (-a) b	&=	(-a + a) b + -(ab)	\\
            (-a) b	&=	(0) b + -(ab)	\\
            (-a) b	&=	0 + -(ab)	\\
            (-a) b	&=	-(ab)
        \end{align*}
    \item $(-a)(-b) = ab$
        \begin{align*}
            (-a)(-b)	&=	(-a)(-b) + ab + -(ab)	\\
            (-a)(-b)	&=	(-a)(-b) + ab + (-a)b	\\
            (-a)(-b)	&=	(-a)(-b + b) + ab	\\
            (-a)(-b)	&=	(-a)(0) + ab	\\
            (-a)(-b)	&=	0 + ab	\\
            (-a)(-b)	&=  ab
        \end{align*}
    \item $ac = bc \wedge  c \neq 0 \Rightarrow a = b$. Let $ac = bc$ and $c \neq 0$.
        \begin{align*}
            ac  &=  bc	\\
            ac - bc &=  0   \\
            (a-b) c &=  0   \\
            a - b   &=  0   \\
            a   &=  b
        \end{align*}
    \item $ab = 0 \Rightarrow a=0 \vee b=0$. Go over all possible values of a. First assume $a = 0$, then $a=0 \vee b=0$ is true. Second assume $a \neq 0$. Then it has inverse $a^{-1}$.
        \begin{align*}
            ab	&=	0	\\
            ab(a^{-1})	&=	0(a^{-1})	\\
            (a \vdot a^{-1}) b	&=	0	\\
            1 b	&=	0	\\
            b	&=	0	\\
        \end{align*} So that $a=0 \vee b=0$ is true since b must be 0.
\end{enumerate}


\question
\begin{align*}
    \frac{a^2 + b^2}{2}	&= \frac{a^2 + b^2 - 2ab + 2ab}{2}  \\
    \frac{a^2 + b^2}{2}	&= \frac{(a-b)^2 + 2ab}{2}  \\
    \frac{a^2 + b^2}{2}	&= \frac{(a-b)^2}{2} + ab  \\
    (a-b)^2 &\geq 0 \\
    \frac{a^2 + b^2}{2}	&\geq ab\\
\end{align*}


\question
Use triangle inequality that $|a - c| \leq |a - b| + |b - c|$.
\begin{align*}
    a   &=  a + 0   \\
    a   &=  a + ((-b) + b)	\\
    a   &=  (a-b) + b	\\
    |a|   &=  |(a-b) + b|	\\
    |(a-b) + b| &\leq |a-b| + |b|   \\
    |a| &\leq |a-b| + |b|   \\
    |a| - |b| &\leq |a-b|   \\
    ||a| - |b|| &\leq |a-b|   \\
\end{align*}


\question
Let the set $[a, b) = S$, and $S \subset R$.

$\inf [a, b) = a$ because
\begin{itemize}
    \item $a$ is a lower bound of $S$ since $a \in \mathbb{R} $ and for all $x \in S, x \geq a$.
    \item Suppose we have a different lower bound $m$ of $S$, then for all $x \in S, x \geq m$. Since $a \in S$, $a \geq m$.
\end{itemize}

$\sup [a, b) = b$ because 
\begin{itemize}
    \item $b$ is an upper bound of $S$ since $b \in \mathbb{R}$ and for all $x \in S, x \le b \Rightarrow x \leq b$.
    \item Suppose we have a different minimal upper bound $M < b$ of $S$, then for all $x \in S, x \leq M$ and $M \in S$. But $b \notin S$, so we can choose $c \in S: M < c < b$. Thus $M$ is not an upper bound, let alone the minimal upper bound.
\end{itemize}


\question
If $T$ is bounded above that means that it has some upper bound $M$ such that $\forall t \in T: t \leq M$. Since $S \subseteq T$ by definition of subset $\forall s \in S: s \in T$. Therefore $\forall s \in S: s \leq M$ so $M$ is an upper bound of $S$ and $S$ is bounded above.

Let $\sup T = m$. Then $\forall t \in T: t \leq m$ and any other upper bound $M$ of $T$ has to be either outside $T$ or equal to $m$, that is, $m \leq M$. Let $\sup S = n$. Note from before $\forall s \in S: s \leq M$ for an upper bound $M$ of $T$. Then we have $\forall s \in S: s \leq m$. Since $n \in S$ we have $n \leq m$, or $\sup S \leq \sup T$.


\question
Note that since $a \leq \sup A$ and $b \leq \sup B$ then $a + b \leq \sup A + \sup B$. Since this is the case for all $a+b$ then $\sup(A+B) \leq \sup A + \sup B$.

Next note that we can choose some $\epsilon > 0$ so that we can represent an $a, b$ as $\sup A - \epsilon, \sup B - \epsilon$. Then $a+b = \sup A + \sup B - 2\epsilon$ so $a+b \geq \sup A + \sup B$. We also have $\sup (A+B) \geq a+b$ so $\sup (A+B) \geq \sup A + \sup B$.

Since we have the two inequalities $\sup(A+B) \leq \sup A + \sup B$ and $\sup (A+B) \geq \sup A + \sup B$, it must be the case that $\sup(A+B) = \sup A + \sup B$.


\question
Define the set $S$ to be ${r \in \mathbb{Q}: r < a}$ for some $a \in \mathbb{R}$. First, $a$ is an upper bound of $S$ since $\forall r \in S: r < a$ by definition of $S$. Suppose we have a different minimal upper bound $M$ of $S$ where $M < a$, then $\forall r \in S: r \leq M$ and $M \in S$. But $a \notin S$. The question is, can we choose a $c \in S: M < c < a$? Note that $M$ and $c$ are rational numbers, while $a$ may be rational or irrational. However, because $\mathbb{Q}$ is dense in $\mathbb{R}$, this means that between \textit{any} two real numbers there will always be a rational number. Thus, there would be a rational $c \in S: M < c < a$, which means that $M$ cannot be the minimal upper bound of $S$. Thus $a$ is the minimal upper bound and therefore $\sup{r \in \mathbb{Q} : r < a} = a$ for all $a \in \mathbb{R}$.








\end{document}
