\documentclass{homework}
\newcommand{\hwname}{Zooey Nguyen}
\newcommand{\hwemail}{zooeyn@ucla.edu}
\newcommand{\hwclass}{Astronomy 82}
\newcommand{\hwtype}{Homework}
\newcommand{\hwnum}{8}
\usepackage{siunitx}
\begin{document}
\maketitle


\question
Because the muon is going at nearly the speed of light, it will experience time dilation and a given time interval in its rest frame will correspond to a much larger time interval from the observer's frame. To the observer, muons will seem to "age" or decay much more slowly than they would if they were at rest alongside the observer, and hit the surface of the Earth before decaying. To the muon, the observer and everything on the Earth will seem to move much more quickly, and it will hit the Earth's surface before 2.2 microseconds passes in its own frame.

If a muon is travelling at 0.995c, then it will travel the following distance.
\begin{align*}
    \Delta t_{obs}	&=	\gamma \Delta t_{muon}	\\
    d    &=	v/\Delta t	\\
    d_{obs} &=  \frac{0.995c}{\Delta t_{obs}}   \\
    d_{obs} &=  \frac{0.995c}{\gamma t_{muon}}   \\
    d_{obs} &=  \frac{0.995c}{\SI{2.2e-5}{\second} / \sqrt{1-(0.995)^2}}   \\
    d_{obs} &=  \frac{0.995 \vdot \SI{3e8}{\metre/\second}}{\SI{2.2e-5}{\second}}   \\
    d_{obs} &=  \boxed{\SI{1.35e13}{\metre}}   \\
\end{align*}


\question
Peak wavelength can be found from blackbody radiation equation and is in the \fbox{infrared} region.
\begin{align*}
    \lambda_{max}	&=	(\SI{2.898e-3}{\metre\kelvin}) / (\SI{3000}{\kelvin})	\\
    \lambda_{max}    &=	\boxed{\SI{966}{\nano\metre}}	\\
\end{align*}
Since we observe these photons as microwave it means that as they travelled towards us the expansion of space expanded them from infrared to microwave photons, or redshifting them. Approximate microwave wavelength is \SI{1}{\milli\metre}
\begin{align*}
    z   &=  \frac{\Delta \lambda}{\lambda_0}	\\
    z   &=  \frac{\SI{1}{\milli\metre} - \SI{966}{\nano\metre}}{\SI{966}{\nano\metre}}	\\
    z   &=  \boxed{1034}	\\
\end{align*}
Use time-dependent Hubble parameter. We guess that the mass-energy density of the universe is at the critical density (flat curvature).
\begin{align*}
    \rho	&\propto    (1+z)^3 \\
    \rho(z)	&=    \rho_0 (1+z)^3 \\
    \rho(z)	&=    (\SI{1.06e-29}{\gram/\centi\metre^3}) (1035)^3 \\
    \rho(z)	&=    \boxed{\SI{1.17e-20}{\gram/\centi\metre^3}} \\
\end{align*}


\question
If one direction is warmer and another is cooler it means we are moving in the warmer direction as we are looking at the slightly earlier part of the universe on that end before the other end.
\begin{align*}
    \SI{3.37}{\milli\kelvin}    &=	T_{hot} - T_{cool}	\\
    T   &\propto    1+z	\\
    T   &=    (\SI{2.728}{\kelvin})1+z	\\
    \SI{3.37}{\milli\kelvin}    &= (\SI{2.728}{\kelvin})(\Delta z) \\
    \Delta z    &=  0.0012   \\
    \Delta v/c  &=  0.0012 \\
    \Delta v    &=  \boxed{\SI{3.6e5}{\metre/\second}}
\end{align*}
The motion may be caused by local gravitational pulls of other galaxies or our supercluster.


\question
Let's calculate the relative velocity from us.
\begin{align*}
    1 + z	&=	\sqrt{\frac{1 + v/c}{1 - v/c}} \\
    6.25	&=	\frac{1 + v/c}{1 - v/c} \\
    6.25 - 6.25 v/c    &=	1 + v/c	\\
    v   &=  0.72c
\end{align*}
So we know that the supernova is moving away from us at a relative velocity of $0.72c$. This means that its photons will be lengthened in space by its movement away from us and redshifted, plus there is time dilation in the time interval we observe those photons from the supernova coming at us. So not only will the lightcurve be flatter, it will also be wider.


\question
Time it takes for the ant to reach pole from equator.
\begin{align*}
    d\theta    &=	\frac{v}{R(t)} dt	\\
    R(t)    &=	\SI{10000}{\milli\metre} + (\SI{1}{\milli\metre/\second})t	\\
    d\theta    &=	\frac{v}{\SI{10000}{\milli\metre} + (\SI{1}{\milli\metre/\second})t} dt	\\
    \int_0^{\pi/2} d\theta  &=  v \int_0^T \frac{dt}{\SI{10000}{\milli\metre} + (\SI{1}{\milli\metre/\second})t}    \\
    \pi/2   &=  v(\si{\second/\milli\metre}) \ln \frac{\SI{10000}{\milli\metre} + (\SI{1}{\milli\metre/\second})T}{\SI{10000}{\milli\metre}}  \\
    \exp{\frac{\pi}{2v(\si{\second/\milli\metre})}}    &=  \frac{\SI{10000}{\milli\metre} + (\SI{1}{\milli\metre/\second})T}{\SI{10000}{\milli\metre}}  \\
    T    &=	\boxed{\frac{(\SI{10000}{\milli\metre}) \exp{\frac{\pi}{2v(\si{\second/\milli\metre})}} - \SI{10000}{\milli\metre}}{\SI{1}{\milli\metre/\second}}}	\\
\end{align*}
For the ant travelling at \SI{5}{\milli\metre/\second} we get the following total time.
\begin{align*}
    T    &=	\frac{(\SI{10000}{\milli\metre}) \exp{\frac{\pi}{2(\SI{5}{\milli\metre/\second})(\si{\second/\milli\metre})}} - \SI{10000}{\milli\metre}}{\SI{1}{\milli\metre/\second}}	\\
    T    &=	\frac{(\SI{10000}{\milli\metre}) \exp{\frac{\pi}{10}} - \SI{10000}{\milli\metre}}{\SI{1}{\milli\metre/\second}}	\\
    T    &=	\frac{\SI{3691}{\milli\metre}}{\SI{1}{\milli\metre/\second}}	\\
    T   &=  \boxed{\SI{3.7e3}{\second}}
\end{align*}
Original distance between equator and the pole.
\begin{align*}
    L_0	&=	\frac{\pi R_0}{2}	\\
    L_0	&=	\frac{\pi (\SI{10000}{\milli\metre})}{2}	\\
    L_0	&=	\boxed{\SI{15.7}{\metre}}	\\
\end{align*}
Final distance between equator and the pole.
\begin{align*}
    L_f	&=	\frac{\pi R_f}{2}	\\
    L_f	&=	\frac{\pi [(\SI{10000}{\milli\metre}) + (\SI{1}{\milli\metre/\second})(\SI{3.7e3}{\second})]}{2}	\\
    L_f	&=	\frac{\pi [(\SI{10000}{\milli\metre}) + (\SI{1}{\milli\metre/\second})(\SI{3.7e3}{\second})]}{2}	\\
    L_f	&=	\boxed{\SI{21.5}{\metre}}	\\
\end{align*}
Total distance the ant walked. I forgot to keep all the digits through my calculations sorry but this looks like it should be exactly halfway between the original and ending distances to the equator (18.6 meters is the midpoint of the previous two answers).
\begin{align*}
    d	&=	vT	\\
    d	&=	(\SI{5}{\milli\metre/\second})(\SI{3691}{\second})	\\
    d   &=  \SI{18500}{\milli\metre}    \\
    d   &=  \boxed{\SI{18.5}{\metre}}
\end{align*}

\end{document}
