\documentclass{homework}
\newcommand{\hwname}{Zooey Nguyen}
\newcommand{\hwemail}{zooeyn@ucla.edu}
\newcommand{\hwclass}{Physics 115A}
\newcommand{\hwtype}{Homework}
\newcommand{\hwnum}{5}
\begin{document}
\maketitle

\question
Expectation values of harmonic oscillator.
\begin{align*}
	\ev{x}
		&=	\sqrt{\frac{\hbar}{2m\omega}} \int_{-\infty}^{\infty}	\Psi_n^*(a_+ \Psi_n + a_-) \Psi_n \dd{x}
		=	\sqrt{\frac{\hbar}{2m\omega}} \int_{-\infty}^{\infty}	\left( \sqrt{n+1}\Psi_n^*\Psi_{n+1} + \sqrt{n}\Psi_n^*\Psi_{n-1}	\right) 	\dd{x}
		=	\boxed{0} \\
	\ev{p}
		&=	m \dv{\ev{x}}{t}
		= \boxed{0}	\\
	\ev{x^2}
		&=	\frac{\hbar}{2m\omega} \int_{-\infty}^{\infty}	\Psi_n^* (a_+^2 + a_-^2 + a_+a_- + a_-a_+) \Psi_n	\dd{x}	\\
		&=	\frac{\hbar}{2m\omega} \int_{-\infty}^{\infty}	\left[	\sqrt{(n+1)(n+2)}\Psi_n^*\Psi_{n+2} + n\Psi_n^*\Psi_n + (n+1)\Psi_n^*\Psi_n + \sqrt{(n-1)n} \Psi_n^*\Psi_{n-2} \right] 	\dd{x}	\\
		&=	\frac{\hbar}{2m\omega} \int_{-\infty}^{\infty}	(2n+1) |\Psi_n|^2	\dd{x}
		=	\frac{(2n+1)\hbar}{2m\omega}
		=	\boxed{(n+\frac{1}{2}) \frac{\hbar}{m\omega}}\\
	\ev{p^2}
		&=	-\frac{m\hbar\omega}{2} \int_{-\infty}^{\infty}	\Psi_n^* (a_+^2 + a_-^2 + a_+a_- + a_-a_+) \Psi_n	\dd{x}
		=	\boxed{(n+\frac{1}{2}) m\hbar\omega}
\end{align*}
Uncertainty principle check.
\begin{align*}
	\sigma_x \sigma_p
		&=	\sqrt{\ev{x^2} - \ev{x}^2} \sqrt{\ev{p^2} - \ev{p}^2}
		=	\sqrt{(n+\frac{1}{2}) \frac{\hbar}{m\omega} - (0)^2} \sqrt{(n+\frac{1}{2}) m\hbar\omega - (0)^2}
		=	(n+\frac{1}{2}) \hbar
		\ge \frac{\hbar}{2}
\end{align*}


\question
Since $E_n = (n+\frac{1}{2}) \hbar \omega_n$, the energies are now $E_n' = (2n+1) \hbar \omega_n$ for $n = 0, 1, 2, \dots$.

Probability of getting $E_n' = \frac{\hbar\omega}{2}$ is \boxed{0} since there is no $n \in $ for which $2n+1 = \mathbb{Z}^+ = \frac{1}{2}$.

Probability of getting $E_n' = \hbar\omega$ is as follows.
\begin{align*}
	C_0
		&=	\int_{-\infty}^{\infty}	\Psi(x,0) \Psi'(x)	\dd{x}
		=	\int_{-\infty}^{\infty}	\left(\frac{m\omega}{\pi\hbar}\right)^{1/4} e^{-\frac{m\omega}{2\hbar}x} \left(\frac{2m\omega}{\pi\hbar}\right)^{1/4} e^{-\frac{m\omega}{\hbar}x} 	\dd{x}	\\
		&=	2^{1/4}	\sqrt{\frac{m\omega}{\pi\hbar}} \int_{-\infty}^{\infty}	e^{-\frac{3m\omega}{2\hbar}x}	\dd{x}
		=	2^{1/4}	\sqrt{\frac{2}{3}}
		=	\boxed{0.9428}
\end{align*}


\newpage
\question
Probability of finding particle outside classical region for ground state in harmonic oscillator.
\begin{align*}
	E_{max}
		&=	\frac{1}{2} m \omega^2 x_{max}^2
		=	\frac{1}{2} \hbar \omega
		\qthen
		x_{max} =	\sqrt{\frac{\hbar}{m\omega}}	\\
	P
		&=	2\sqrt{\frac{m\omega}{\pi\hbar}}	\int_{\sqrt{\frac{\hbar}{m\omega}}}^{\infty}	e^{-\frac{m\omega}{\hbar}x^2}	\dd{x}
		=	2\sqrt{\frac{m\omega}{\pi\hbar}}\sqrt{\frac{\hbar}{m\omega}}	\int_{1}^{\infty}	e^{-u^2}	\dd{u}
		=	\frac{2}{\sqrt{\pi}} \frac{\sqrt{\pi}}{2} (1-erf(1))
		=	\boxed{0.1573}
\end{align*}


\question


\question
Solutions can no longer include even solutions because the wavefunction must go to 0 at $x=0$. Since only odd solutions exist we recover the same energies as the regular harmonic oscillator but only for odd integers, $E_n = (n+\frac{1}{2})\hbar\omega, n = 1, 3, 5, \dots$.

\end{document}