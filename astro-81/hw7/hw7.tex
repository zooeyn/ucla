\documentclass[newpage]{homework}
\newcommand{\hwname}{Zooey Nguyen}
\newcommand{\hwemail}{zooeyn@ucla.edu}
\newcommand{\hwclass}{Astro 81}
\newcommand{\hwtype}{Homework}
\newcommand{\hwnum}{7}
\usepackage{siunitx}
\begin{document}
\maketitle

\question
\begin{alphaparts}
    \questionpart Gravitational binding energy of the Sun and energy usage gives us an estimated lifetime.
        \begin{align*}
            U   &=	\frac{3}{5} \frac{GM_\odot^2}{R}	\\
            U   &=	\frac{3}{5} \frac{\SI{6.67e-11}{\newton\metre^2\per\kilogram^2} \vdot \SI{2e30}{\kilogram}^2}{\SI{6.96e8}{\metre}}	\\
            U   &=	\SI{2.30e41}{\joule}	\\
            U   &=  t_{pred} \vdot L_\odot \\
            t_{pred}    &=	\SI{2.30e41}{\joule} / \SI{3.8e26}{\joule\per\second}	\\
            t_{pred}    &=  \SI{6.05e14}{\second}   \\
            t_{pred}    &=	\boxed{\SI{2e7}{yr}}
        \end{align*}
        This would only last the sun 20 million years, not a few billion.
    \questionpart Approximate the star to be 100\% Hydrogen and constant luminosity till it is all burned up after its lifetime of \num{10e10} years.
        \begin{align*}
            \text{Efficiency}   &=	\frac{E_{actual}}{E_{potential}}	\\
            E_{actual}  &=  t_\odot \vdot L_\odot   \\
                        &=  \SI{10e10}{yr} \vdot \SI{3.8e26}{\joule\per\second}	\\
                        &=  \SI{1.2e45}{\joule}	\\
            E_{potential}   &=  mc^2    \\
                        &=  0.1 \vdot \SI{2e30}{\kilogram} \vdot (\SI{3e8}{\metre\per\second})^2    \\
                        &=  \SI{1.8e46}{\joule} \\
            \text{Efficiency}   &=  \boxed{\SI{6.67}{\percent}}
        \end{align*}
    \questionpart If it has to burn through the remaining 90\% of hydrogen that's 9 times the previous amount of energy to burn through, with 100 times the rate of burning. The new time $t_{\odot,red} = 0.09 t_{\odot,main}$ or \fbox{\SI{9}{\percent}} the main sequence lifetime.
\end{alphaparts}


\question
\begin{alphaparts}
    \questionpart Mean free path can be calculated with expected value of distance to the closest particle given a 3D Poisson distribution of particles. But the equation is
        \begin{align*}
            l_\gamma    &= \frac{1}{\kappa_\gamma \rho}	\\
            l_\gamma    &= \frac{1}{\SI{10}{\metre^2\per\kilogram} \vdot \SI{e5}{\kilogram\per\metre^3}}	\\
            l_\gamma    &= \boxed{\SI{e-6}{\metre}}
        \end{align*}
    \questionpart Same as the above but replace the mass absorption coefficient.
        \begin{align*}
            l_{neutrino}    &= \frac{1}{\kappa_\gamma \rho}	\\
            l_{neutrino}    &= \frac{1}{\SI{e-21}{\metre^2\per\kilogram} \vdot \SI{e5}{\kilogram\per\metre^3}}	\\
            l_{neutrino}    &= \boxed{\SI{e16}{\metre}}
        \end{align*}
    \questionpart Use velocity of the photon with the mean free path.
        \begin{align*}
            ct_\gamma	&=	l_\gamma	\\
            t_\gamma    &=  \frac{\SI{e-6}{\metre}}{\SI{3e8}{\metre\per\second}}    \\
            t_\gamma    &=	\boxed{\SI{3.3e-15}{\second}}
        \end{align*}
    \questionpart Central pressure of a $10M_\odot$ star will be about $0.01 P_\odot$ since central density is proportional to $1/M^2$. Since central pressure decreases, central density will too, and for an ideal gas we have $P = n\rho T$ so $P \propto \rho$ so the new central density is also about $0.01 \rho_\odot$ from what we had before. $l_\gamma \propto 1/\rho$ and $t_\gamma \propto l_\gamma$ so $t_\gamma \propto 1/\rho$ so we get that the new mean free path time is $1/0.01$ or \fbox{100 times} the answer in part c.
\end{alphaparts}


\question
\begin{alphaparts}
    \questionpart Calculate absolute magnitude then luminosity which is in units of Sun luminosity.
        \begin{align*}
            m - M	&=	5 \log R - 5	\\
            7.5 - M   &=	5 \log 1000 - 5	\\
            M   &=  -2.5    \\
            -2.5 \log (L/L_\odot) &=  -2.5    \\
            L   &=  \boxed{10 L_\odot}
        \end{align*}
    \questionpart Mass-luminosity relation is approximately $L \propto M^3$ so if the luminosity is 10 times that of the sun the mass is $10^{1/3}$ times the mass of the sun, or \fbox{$2.15 m_\odot$}.
    \questionpart Since $t_{ms} \propto M/L$ we can get an approximate relation to get its age in relation to the Sun where the Sun's lifetime is $t_\odot = \SI{e10}{yr}$, $t = 2.15 / 10 t_\odot$ which is \fbox{\SI{2.15e9}{yr}}.
\end{alphaparts}

\end{document}
